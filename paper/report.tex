% status: 0
% chapter: IaaS

\title{Apache Mesos and Mesosphere}


\author{Ricky Alan Carmickle Jr}
\affiliation{%
  \institution{Indiana University}
  \streetaddress{School of Informatics and Computing}
  \city{Bloomington} 
  \state{IN} 
  \postcode{47408}
  \country{USA}}
\email{rcarmick@iu.edu}


% The default list of authors is too long for headers}
\renewcommand{\shortauthors}{R. A. Carmickle Jr.}


\begin{abstract}

Apache Mesos is as distributed cluster computing kernel and datacenter operating platform.  Mesos performs container support in a scalable enviromnent by splitting scheduling into a two-level architecture. Applications running on Mesos are containerized separately from the framework handling infrastructure scheduling operations. Mesosphere performs resource consolidation, resource isolation, and storage capabilities in a scalable system as it runs distributed containerized software. Mesosphere is a commercial platform built on Apache Mesos.


\end{abstract}

\keywords{hid404, Apache Mesos, Mesosphere, container, cluster}


\maketitle


\section{Introduction}

\TODO{All urls must be moved to a real citation. Papers do not include citations.} 

The open source Apache Mesos kernel can downloaded at

\url{http://mesos.apache.org/downloads/}

The repository for Mesos is hosted at 

\url{https://github.com/mesosphere}

Documentation for Mesosphere can be found at

\url{https://docs.mesosphere.com/1.11/installing/oss/custom/}

Mesosphere can be installed at

\url{https://dcos.io/install/?_ga=2.232557215.1886022105.1520970766-28104652.1516738235}

Apache Mesos is a framework for cluster computing management for large-scale data applications. Development on Mesos began in 2009 at UC Berkeley and it became a top level Apache product on 2013~\cite{hid-sp18-404-Kakadia2015}. Mesos is a data center kernel which provides views of and access to all resources on all machines in a cloud computing data center. The key feature of Mesos is a two-level scheduler which can manage a diverse array of applications with minimal heaving by the Mesos interface. It uses container-based isolation as opposed to machine virtualization, which dramatically lowers the overhead for applications. Mesos operates with one active master framework, elected by Apache Zookeeper~\cite{hid-sp18-404-Winder15}, which mediates between slave frameworks and resources. Mesos is a competing service to Kubernetes and Docker Swarm at job scheduling and resource management~\cite{hid-sp18-404-Ivanovic2016}, but can be used alongside both those cluster management options. 

\section{Architecture}

Mesos is a solution to the challenge of scaling data center needs while maintaining usability. The needs of both developers and infrastructure operators are met through the abstraction between hardware and applications by the Mesos kernel. Developers are able to focus on the logic of applications without concern for the hardware their application must run on. Operators are free from concern about the allocation of resources and scalability. An effective data center kernel must meet the key requirements of scalability in the number of machines used, flexibility with the number of compatible frameworks, maintainability as infrastructure changes, dynamic adaptation to new available resources, and fairness in resource allocation to different users. The general Mesos cluster is composed of Mesos masters, Mesos slaves, frameworks, communication, and auxiliary services.

The Mesos master is the mechanism which allocates cluster resources between frameworks without offering either too few or too many resources. 
Mesos slaves are the mechanisms responsible for using resources and performing tasks from frameworks. Frameworks are the applications which perform a wide array of tasks and are built from a scheduler component and an executor component. Communication within Mesos happens with the libprocess~\cite{hid-sp18-404-Kakadia2015} library with a protocol similar to HTTP. The auxiliary services are not part of Mesos but are necessary for an effective deployment of a Mesos cluster. The most common auxiliary services are shared file systems such as Amazon S3, a master election service such as Apache Zookeeper, and services which connect Mesos-hosted frameworks to one another~\cite{hid-sp18-404-Denman2017}.

The Mesos two-level scheduler has a simple workflow. The framework is registered with the master. The slave makes a resource offer to the master which uses it's allocation module to decide which framework should receive the resources. The framework scheduler receives the resource offer and decides whether the offer is acceptable for it's required tasks. If the offer is accepted by the framework, the slave will then allocate the resources and execute the framework's tasks. The framework executor will notify the scheduler when the task has been completed. This process repeats for as long as frameworks have tasks and are registered with the master for resource offers. 

Frameworks are containerized by an internal container mechanism within Mesos. Containers are based on both process and control group isolation. 

\section{Installation and Setup}
Mesos is best installed from the command line on most operating systems and offers test frameworks in languages including C++, Java, and Python~\cite{hid-sp18-404-Kakadia2015}. The Mesos kernel can be launched as a single node from Linux or OSx~\cite{hid-sp18-404-Kakadia2015}. Multi-node deployment is performed through SSH reliant scripts included in Mesos. 


\section{Applications}

Mesos is compatible with a majority of big data and cloud computing software. It is therefore compatible with the most important big data frameworks. The most common applications deployed on Mesos are Apache Spark, Kafka, Elasticsearch, Cassandra NoSQL, and Hadoop~\cite{hid-sp18-404-Yegulalp2016}. 

Apache Spark is a cluster computing framework. Mesos can act as the cluster manager for Spark and allows multiple instances of Spark. Because instances of Spark must share resources with other frameworks when running on Mesos, tuning on the Spark implementation is necessary depending on the scale and needs of a given project. 

Apache Kafka is ``a distributed, high-throughput, low-latency publish-subscribe messaging system''~\cite{hid-sp18-404-Narkhede2015}. Kafka was one of the first other Apache projects to be incorporated with Mesos. Any version of Kafka can run on Mesos and stream simulations can be run in real-time since Mesos allows multiple instances of Kafka to run simultaneously. Kafka commands are integrated into the Mesos scheduler so the data streaming and messaging can be managed directly from Mesos REST API. 

Elasticsearch is a distributed text search and analytics engine which creates a very efficient combination with Mesos when an already fast search function can be duplicated across nodes with Mesos to run multiple search functions together~\cite{hid-sp18-404-Vanderzyden2015}. 

Cassandra NoSQL is the popular NoSQL database which offers extensive integration for cluster datacenters.  Mesos allows for the installation of Cassandra so that NoSQL queries can be run through the Cassandra Query Language shell. 

Apache Hadoop's MapReduce version 2 does not run on Mesos, so installation of Hadoop must be done with MapReduce1. Multiple versions of Hadoop can be run on Mesos container clusters. 

Development of a Mesos application is a process of creating a development environment, adding framework scheduler and launcher, writing the executors, then compiling and installing the framework on the cluster wherever necessary.  

\section{Analytics}

The primary analytics tools for big data on Mesos are Hadoop, Spark SQL, Hive, Shark, and NoSQL databases like Cassandra. Analytics tools which operate on Mesos are applied to each of the Batch, Speed, and Serving layers of the Lambda architecture~\cite{hid-sp18-404-Kakadia2015}. Analytic frameworks on Mesos are designed to meet the needs of current data applications to process and analyze data in real-time instead of periodical or offline processing. Spark, Kafka, and Hadoop are the earliest and most popular analytics frameworks compatible on Mesos, but many more frameworks have been in development since 2016. 

\section{Mesosphere}

Mesosphere a proprietary product offered by the company of the same name. Mesosphere performs all the tasks described in Apache Mesos in a user-friendly interface and with a dedicated team who respond to the needs of client companies and focus on incorporation of Mesos with other services like Kubernetes, Docker, and emerging AI machine learning tools~\cite{hid-sp18-404-Mae2018}. Mesosphere contributes to both Apache Mesos and Mesosphere DC/OS (datacenter scale operating center)

\section{Conclusion}

Apache Mesos and Mesosphere are powerful cluster management tools which have proven effective with large scale data applications like Airbnb, Twitter, and Uber. Mesos has emerged as one of several options for container and cluster management as big data applications have demanded resource management which optimized resource allocation. 


\begin{acks}

  The authors would like to thank Dharmesh Kakadia, the author of Apache Mesos Essentials, for the very informative text on Apache Mases

\end{acks}

\bibliographystyle{ACM-Reference-Format}
\bibliography{report} 

